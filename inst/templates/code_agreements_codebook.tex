% Options for packages loaded elsewhere
\PassOptionsToPackage{unicode}{hyperref}
\PassOptionsToPackage{hyphens}{url}
%
\documentclass[
]{article}
\usepackage{lmodern}
\usepackage{amssymb,amsmath}
\usepackage{ifxetex,ifluatex}
\ifnum 0\ifxetex 1\fi\ifluatex 1\fi=0 % if pdftex
  \usepackage[T1]{fontenc}
  \usepackage[utf8]{inputenc}
  \usepackage{textcomp} % provide euro and other symbols
\else % if luatex or xetex
  \usepackage{unicode-math}
  \defaultfontfeatures{Scale=MatchLowercase}
  \defaultfontfeatures[\rmfamily]{Ligatures=TeX,Scale=1}
\fi
% Use upquote if available, for straight quotes in verbatim environments
\IfFileExists{upquote.sty}{\usepackage{upquote}}{}
\IfFileExists{microtype.sty}{% use microtype if available
  \usepackage[]{microtype}
  \UseMicrotypeSet[protrusion]{basicmath} % disable protrusion for tt fonts
}{}
\makeatletter
\@ifundefined{KOMAClassName}{% if non-KOMA class
  \IfFileExists{parskip.sty}{%
    \usepackage{parskip}
  }{% else
    \setlength{\parindent}{0pt}
    \setlength{\parskip}{6pt plus 2pt minus 1pt}}
}{% if KOMA class
  \KOMAoptions{parskip=half}}
\makeatother
\usepackage{xcolor}
\IfFileExists{xurl.sty}{\usepackage{xurl}}{} % add URL line breaks if available
\IfFileExists{bookmark.sty}{\usepackage{bookmark}}{\usepackage{hyperref}}
\hypersetup{
  pdftitle={Codebook for qID generated by code\_agreements() function},
  hidelinks,
  pdfcreator={LaTeX via pandoc}}
\urlstyle{same} % disable monospaced font for URLs
\usepackage[margin=1in]{geometry}
\usepackage{longtable,booktabs}
% Correct order of tables after \paragraph or \subparagraph
\usepackage{etoolbox}
\makeatletter
\patchcmd\longtable{\par}{\if@noskipsec\mbox{}\fi\par}{}{}
\makeatother
% Allow footnotes in longtable head/foot
\IfFileExists{footnotehyper.sty}{\usepackage{footnotehyper}}{\usepackage{footnote}}
\makesavenoteenv{longtable}
\usepackage{graphicx,grffile}
\makeatletter
\def\maxwidth{\ifdim\Gin@nat@width>\linewidth\linewidth\else\Gin@nat@width\fi}
\def\maxheight{\ifdim\Gin@nat@height>\textheight\textheight\else\Gin@nat@height\fi}
\makeatother
% Scale images if necessary, so that they will not overflow the page
% margins by default, and it is still possible to overwrite the defaults
% using explicit options in \includegraphics[width, height, ...]{}
\setkeys{Gin}{width=\maxwidth,height=\maxheight,keepaspectratio}
% Set default figure placement to htbp
\makeatletter
\def\fps@figure{htbp}
\makeatother
\setlength{\emergencystretch}{3em} % prevent overfull lines
\providecommand{\tightlist}{%
  \setlength{\itemsep}{0pt}\setlength{\parskip}{0pt}}
\setcounter{secnumdepth}{-\maxdimen} % remove section numbering

\title{Codebook for qID generated by code\_agreements() function}
\author{}
\date{\vspace{-2.5em}}

\begin{document}
\maketitle

The function creates a new column of qID based on information extracted
from the treaty title column of a dataset. This allows to assign unique
ID to treaties and should facilitate the analysis of multiple datasets
and the detection of duplicates or treaties from the same family. The
function works through different steps.

\hypertarget{how-to-read-a-qid}{%
\subsubsection{How to read a qID}\label{how-to-read-a-qid}}

A qID is a meaningful shorthand ID created from a combination of
elements extracted from the agreement title an date. The qID allows
users to identify bilateral treaties, treaty topics, type, linkage and
more.

\begin{longtable}[]{@{}lll@{}}
\toprule
\begin{minipage}[b]{0.30\columnwidth}\raggedright
\textbf{Type}\strut
\end{minipage} & \begin{minipage}[b]{0.30\columnwidth}\raggedright
\textbf{Pasting}\strut
\end{minipage} & \begin{minipage}[b]{0.30\columnwidth}\raggedright
\textbf{Pasting}\strut
\end{minipage}\tabularnewline
\midrule
\endhead
\begin{minipage}[t]{0.30\columnwidth}\raggedright
Bilateral\strut
\end{minipage} & \begin{minipage}[t]{0.30\columnwidth}\raggedright
WAT\_19060521A\_MEX-USA\strut
\end{minipage} & \begin{minipage}[t]{0.30\columnwidth}\raggedright
topic\_uID, type\_parties\strut
\end{minipage}\tabularnewline
\begin{minipage}[t]{0.30\columnwidth}\raggedright
Bilateral + Protocol\strut
\end{minipage} & \begin{minipage}[t]{0.30\columnwidth}\raggedright
WAT\_18840921E3\_18840921\_BAD-CHE\strut
\end{minipage} & \begin{minipage}[t]{0.30\columnwidth}\raggedright
topic\_uID,type(number)\_linkage\_parties\strut
\end{minipage}\tabularnewline
\begin{minipage}[t]{0.30\columnwidth}\raggedright
Multilateral\strut
\end{minipage} & \begin{minipage}[t]{0.30\columnwidth}\raggedright
SPE\_20050205A\strut
\end{minipage} & \begin{minipage}[t]{0.30\columnwidth}\raggedright
topic\_uID, type\strut
\end{minipage}\tabularnewline
\begin{minipage}[t]{0.30\columnwidth}\raggedright
Multi + Protocol\strut
\end{minipage} & \begin{minipage}[t]{0.30\columnwidth}\raggedright
OTH\_19920817E2\_20070817\strut
\end{minipage} & \begin{minipage}[t]{0.30\columnwidth}\raggedright
topic\_uID,type(number)\_linkage\strut
\end{minipage}\tabularnewline
\begin{minipage}[t]{0.30\columnwidth}\raggedright
Known treaties\strut
\end{minipage} & \begin{minipage}[t]{0.30\columnwidth}\raggedright
UNCLOS\_19821210\strut
\end{minipage} & \begin{minipage}[t]{0.30\columnwidth}\raggedright
abbreviation\_uID\strut
\end{minipage}\tabularnewline
\begin{minipage}[t]{0.30\columnwidth}\raggedright
Protocol of known treaties\strut
\end{minipage} & \begin{minipage}[t]{0.30\columnwidth}\raggedright
MARPOL\_19731102\_19900316E\strut
\end{minipage} & \begin{minipage}[t]{0.30\columnwidth}\raggedright
abbreviation\_linkage\_uID,type(number)\strut
\end{minipage}\tabularnewline
\bottomrule
\end{longtable}

A bilateral treaty that is an agreement will have the following qID:
``WAT\_19060521A\_MEX-USA''. This is a combination of the topic (WAT)
and date (19060521) with the type(A) and the parties to agreement
(MEX-USA). A bilateral treaty that is any other type than an agreement
(e.g.~protocol, amendments) will have the qID under this format:
``WAT\_18840921E3\_BAD-CHE\_18830703''. The topic (WAT), the date of
signature of the amendment (18840921), the type (E) which refers to
amendment, the number of amendment (3), the parties (BAD-CHE) and
finally the signature date of the ``mother'' treaty (18830703).

\begin{longtable}[]{@{}lll@{}}
\toprule
\textbf{Type} & \textbf{Pasting} & \textbf{Pasting}\tabularnewline
\midrule
\endhead
Multilateral & SPE\_20050205A & topic\_uID, type\tabularnewline
Multi + Protocol & OTH\_19920817E2\_20070817 &
topic\_uID,type(number)\_linkage\tabularnewline
\bottomrule
\end{longtable}

A multilateral treaty that is an agreement will have the following qID:
``SPE\_20050205A'' which indicates the topic (SPE), signature date of
the agreement (20050205), and the type (A). A multilateral treaty that
is not an agreement will have this qID form:
``OTH\_19920817E2\_20070817''. This represents the topic (OTH), the
signature date of the amendment (19920817), the type (E), the number of
amendment (2) with the linkage number (20070817) which is the signature
date of the ``mother'' treaty.

\begin{longtable}[]{@{}lll@{}}
\toprule
\begin{minipage}[b]{0.30\columnwidth}\raggedright
\textbf{Type}\strut
\end{minipage} & \begin{minipage}[b]{0.30\columnwidth}\raggedright
\textbf{Pasting}\strut
\end{minipage} & \begin{minipage}[b]{0.30\columnwidth}\raggedright
\textbf{Pasting}\strut
\end{minipage}\tabularnewline
\midrule
\endhead
\begin{minipage}[t]{0.30\columnwidth}\raggedright
Known treaties\strut
\end{minipage} & \begin{minipage}[t]{0.30\columnwidth}\raggedright
UNCLOS\_19821210\strut
\end{minipage} & \begin{minipage}[t]{0.30\columnwidth}\raggedright
abbreviation\_uID\strut
\end{minipage}\tabularnewline
\begin{minipage}[t]{0.30\columnwidth}\raggedright
Protocol of known treaties\strut
\end{minipage} & \begin{minipage}[t]{0.30\columnwidth}\raggedright
MARPOL\_19731102\_19900316E\strut
\end{minipage} & \begin{minipage}[t]{0.30\columnwidth}\raggedright
abbreviation\_linkage\_uID,type(number)\strut
\end{minipage}\tabularnewline
\bottomrule
\end{longtable}

Famous multilateral treaties have a simplified qID with an knowm
abbreviation instead of the topic. For example, the United Nations
Convention On The Law Of The Sea will have the following qID:
``UNCLOC\_19821210A'' which is the known abbreviation (UNCLOS), the
signature date (19821210) with the type (A). The protocols or amendments
of the known treaties will have this qID:
``MARPOL\_19890317E2\_19731102''. It indicates the known abbreviation
(MARPOL), the signature date of this specific amendment (19890317), the
type of treaty (E), its number (2) and the signature date of the
``mother'' treaty (19731102).

There are also several steps to how the code\_agreements() function
work. Each step relies on a helper function to extract certain elements
that later will be joined together to form a qID. Each of the functions
can also be used as a standalone function to generate certain types of
information. For more information on the code\_agreements functions and
the helper functions, please also run \texttt{?code\_agreements()}.

\#\#\#Code Agreements Codebook

\hypertarget{parties}{%
\subsubsection{Parties}\label{parties}}

The first element extracted from the title is the state that is part to
the treaty. The function \texttt{code\_states()} from \texttt{qStates}
is used. It returns 3 digit ISO codes for states.

\hypertarget{unique-number-uid}{%
\subsubsection{Unique number (uID)}\label{unique-number-uid}}

Each treaty contains a unique number in their qID which refer to the
signature date. The number has therefore eight numbers that are under
the YYYYMMDD format and is called uID.

\hypertarget{type}{%
\subsubsection{Type}\label{type}}

The function detects the type of treaty and assign a letter that will be
included in the final ID. If the treaty is an agreement, the \textbf{A}
will not appear. For all the others, like protocols, amendments, the
assigned letter (respectively P, E, etc) appears in the qID. For
amendments or protocol, their ordering number is also extracted from the
title to be included in the qID.

\hypertarget{topic}{%
\subsubsection{Topic}\label{topic}}

Based on a list of key terms, the topics that are mentioned in the title
will appear in the qID. Some of the topics covered are

\textbf{Example}

\begin{longtable}[]{@{}ll@{}}
\toprule
\begin{minipage}[b]{0.47\columnwidth}\raggedright
\textbf{Words}\strut
\end{minipage} & \begin{minipage}[b]{0.47\columnwidth}\raggedright
\textbf{Topic}\strut
\end{minipage}\tabularnewline
\midrule
\endhead
\begin{minipage}[t]{0.47\columnwidth}\raggedright
Waste,disposal,pollut, toxic, hazard\strut
\end{minipage} & \begin{minipage}[t]{0.47\columnwidth}\raggedright
WAS\strut
\end{minipage}\tabularnewline
\begin{minipage}[t]{0.47\columnwidth}\raggedright
species,habitat,ecosystems,biological diversity,genetic
resources,biosphere,birds,locusts\strut
\end{minipage} & \begin{minipage}[t]{0.47\columnwidth}\raggedright
SPE\strut
\end{minipage}\tabularnewline
\begin{minipage}[t]{0.47\columnwidth}\raggedright
air,atmos,climate,outer space,ozone,emissions\strut
\end{minipage} & \begin{minipage}[t]{0.47\columnwidth}\raggedright
AIR\strut
\end{minipage}\tabularnewline
\begin{minipage}[t]{0.47\columnwidth}\raggedright
water,freshwater,river,rhine,hydro,basin,drought,ocean,shelf,Atlantic,Lake\strut
\end{minipage} & \begin{minipage}[t]{0.47\columnwidth}\raggedright
WAT\strut
\end{minipage}\tabularnewline
\begin{minipage}[t]{0.47\columnwidth}\raggedright
Soil,Wetland,Desert,Erosion,Land,Archipelago\strut
\end{minipage} & \begin{minipage}[t]{0.47\columnwidth}\raggedright
SOI\strut
\end{minipage}\tabularnewline
\begin{minipage}[t]{0.47\columnwidth}\raggedright
nature,environment,biodiversity,flora,plant,fruit,vegetable,seed,forest,tree,conservation,preservation\strut
\end{minipage} & \begin{minipage}[t]{0.47\columnwidth}\raggedright
BIO\strut
\end{minipage}\tabularnewline
\begin{minipage}[t]{0.47\columnwidth}\raggedright
fish,salmon,herring,tuna,aquaculture,mariculture,molluscs,whaling\strut
\end{minipage} & \begin{minipage}[t]{0.47\columnwidth}\raggedright
FIS\strut
\end{minipage}\tabularnewline
\begin{minipage}[t]{0.47\columnwidth}\raggedright
agricultur,food,livestock,crop,irrigation,cattle,meat,farm,cultivate\strut
\end{minipage} & \begin{minipage}[t]{0.47\columnwidth}\raggedright
AGR\strut
\end{minipage}\tabularnewline
\begin{minipage}[t]{0.47\columnwidth}\raggedright
culture,scien,techno,trade,research,exploration,navigation,data,information\strut
\end{minipage} & \begin{minipage}[t]{0.47\columnwidth}\raggedright
BOT\strut
\end{minipage}\tabularnewline
\begin{minipage}[t]{0.47\columnwidth}\raggedright
energy,nuclear,oil,mining,gas,hydro,power\strut
\end{minipage} & \begin{minipage}[t]{0.47\columnwidth}\raggedright
NUC\strut
\end{minipage}\tabularnewline
\begin{minipage}[t]{0.47\columnwidth}\raggedright
accidents\strut
\end{minipage} & \begin{minipage}[t]{0.47\columnwidth}\raggedright
ACC\strut
\end{minipage}\tabularnewline
\begin{minipage}[t]{0.47\columnwidth}\raggedright
chemicals, pesticides,toxin,Lead\strut
\end{minipage} & \begin{minipage}[t]{0.47\columnwidth}\raggedright
CHE\strut
\end{minipage}\tabularnewline
\begin{minipage}[t]{0.47\columnwidth}\raggedright
climate\strut
\end{minipage} & \begin{minipage}[t]{0.47\columnwidth}\raggedright
CLC\strut
\end{minipage}\tabularnewline
\begin{minipage}[t]{0.47\columnwidth}\raggedright
noise\strut
\end{minipage} & \begin{minipage}[t]{0.47\columnwidth}\raggedright
NOI\strut
\end{minipage}\tabularnewline
\begin{minipage}[t]{0.47\columnwidth}\raggedright
disease,diseases\strut
\end{minipage} & \begin{minipage}[t]{0.47\columnwidth}\raggedright
DIS\strut
\end{minipage}\tabularnewline
\begin{minipage}[t]{0.47\columnwidth}\raggedright
resource,resources,timber,antartic,fur,Ivory,Horn\strut
\end{minipage} & \begin{minipage}[t]{0.47\columnwidth}\raggedright
RES\strut
\end{minipage}\tabularnewline
\bottomrule
\end{longtable}

\hypertarget{linkages-between-agreements}{%
\subsubsection{Linkages between
Agreements}\label{linkages-between-agreements}}

Treaties from the same family can be detected through by removing
predicatble words that are added to treaty titles (e.g.~amendment,
protocol, meeting) and identifying duplicates based on the core words
used to refer to main agreement. The YYYYMMDD number assigned to the
``mother'' treaty is used for all the other treaties deriving from it as
the last digits in their qID.

\hypertarget{known-agreements}{%
\subsubsection{Known Agreements}\label{known-agreements}}

Some treaties already have known abbreviation (UNCLOS, UNFCCC, MARPOL,
etc). The qID of these treaties contain their abbreviation instead of
the topic.

\end{document}
